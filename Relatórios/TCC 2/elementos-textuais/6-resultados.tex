%
% Documento: Resultados 
%

\chapter{Resultados}
\label{cap:resultados}

Neste capítulo são apresentados e analisados os resultados obtidos com a realização dos experimentos de ordenação no ambiente Hadoop.

A primeira parte dos experimentos consistiu em testes para conhecer o framework utilizado, o ambiente de execução e validar o funcionamento dos algoritmos de ordenação a serem comparados. A segunda parte dos experimentos iniciou-se após essa fase de reconhecimento, com a realização dos testes para avaliação de desempenho dos algoritmos Ordenação por Amostragem e Quicksort executados de acordo com o planejamento descrito anteriormente na Seção \ref{sec:experimentos}. 


%Esses experimentos foram  
%, e o reproduzir os resultados já encontrados no trabalho de referência: testes de ordenação com os \textit{benchmarks TeraSort} e \textit{Sort}, e com o algoritmo Ordenação por Amostragem em distribuição uniforme. 

A execução inicial de testes com os algoritmos distribuídos com o próprio ambiente Hadoop são fundamentais para a compreensão do funcionamento do ambiente de execução, incluindo a conexão entre as máquinas que compõem o \textit{cluster}.  
A importância desses testes consiste no fato de tais  aplicações serem conhecidas e consolidadas no ambiente Hadoop, e fornecerem exemplos de aplicações de ordenação desenvolvidas no modelo MapReduce.

O \textit{TeraSort} consiste de três algoritmos, que são responsáveis pela geração dos dados, ordenação e validação, conforme descrito na seção \ref{sec:benchmarks}.
Os testes com o \textit{TeraSort} foram feitos em duas máquinas. Foram gerados pelo \textit{TeraGen} dois arquivos contendo 50 mil linhas cada e o algoritmo foi executado 10 vezes.

O \textit{Sort} é um dos \textit{benchmarks}  de ordenação de dados mais conhecidos para Hadoop. Para os testes realizados com esse algoritmo foram utilizados dados gerados pelo algoritmo \textit{RandomWriter}. Para cada máquina do \textit{cluster}, foram escritos 10 arquivos de 1GB cada em formato binário, totalizando 10GB.

Os testes realizados inicialmente com os \textit{benchmarks} \textit{TeraSort} e \textit{Sort} permitiram um melhor entendimento sobre o funcionamento do \textit{framework} e dos algoritmos de ordenação nesse ambiente, que são objeto de estudo do presente trabalho. 

Os testes feitos com o algoritmo Ordenação por Amostragem tinham como objetivo reproduzir os
resultados encontrados no trabalho feito por Pinhão (2011), e gerar resultados 
que serão utilizados na comparação de desempenho dos dois algoritmos. 


Antes da execução dos experimentos de avaliação de desempenho, foram realizados diversos
testes de verificação com o objetivo de comprovar o funcionamento do algoritmo Quicksort em cenários variados,
com diferentes entradas, quantidade de dados e máquinas.
Os testes envolveram arquivos compostos de valores aleatórios, de valor único ou de dois valores,
e em todos os casos foi possível comprovar o correto funcionamento do algoritmo.
 



O resultado dos experimentos está separado de acordo com o tipo de teste realizado, com variação do número de partições,
da distribuição dos dados, da quantidade de dados ordenada e da quantidade de máquinas utilizadas. 


% --------------------------------------------------------
% ----------------- ESTABILIDADE -------------------------
% --------------------------------------------------------
\newpage
\section{Estabilidade do Sistema}

Os testes de estabilidade avaliam a variabilidade do tempo de resposta na execução do algoritmo, que é diretamente afetado pelo tamanho das partições formadas. 

\begin{defaultFigure}{0.75\textwidth}{graficos/EstabilidadeTempo.pdf}
{Gráfico do tempo para ordenação de conjuntos de 10$^8$ dados em 5 máquinas}
\label{fig:EstabilidadeTempo}
\end{defaultFigure}

O tamanho das partições dos dois algoritmos separados, em escalas diferentes.
\begin{defaultFigure}{0.75\textwidth}{graficos/EstabilidadeParticoes1.pdf}
{Partições obtidas na ordenação de conjuntos de 10$^8$ dados em 5 máquinas}
\label{fig:EstabilidadeParticoes}
\end{defaultFigure}


% --------------------------------------------------------
% ----------------- PARTIÇÕES ----------------------------
% --------------------------------------------------------
\newpage
\section{Variando o Número de Partições}

<INCLUIR DESCRIÇÃO GRÁFICO TEMPO>
 
\begin{defaultFigure}{0.75\textwidth}{graficos/ParticoesTempo.pdf}
{Gráfico do tempo para ordenação de conjuntos de 10$^8$ dados em 5 máquinas com diferentes números
de partições}
\label{fig:ParticoesTempo}
\end{defaultFigure}

% --------------------------------------------------------
% ----------------- DISTRIBUIÇÃO -------------------------
% --------------------------------------------------------
\newpage
\section{Variando a Distribuição de Dados}

<TEXTO> 

<DESCRIÇÃO GRÁFICO TEMPO>
\begin{defaultFigure}{0.75\textwidth}{graficos/DistribuicaoTempo.pdf}
{Gráfico dos tempos médios para ordenação de 10$^8$ dados em 5 máquinas}
\label{fig:DadosTempos}
\end{defaultFigure}

<TABELA TEMPO>

<DESCRIÇÃO GRÁFICO PARTIÇÕES>
% \begin{defaultFigure}{0.75\textwidth}{graficos/DistribuicaoTempo.pdf}
% {Gráfico dos tempos médios para ordenação de 10$^8$ dados em 5 máquinas}
% \label{fig:DadosTempos}
% \end{defaultFigure}

<TABELA PARTIÇÕES>

% --------------------------------------------------------
% -------------------- DADOS -----------------------------
% --------------------------------------------------------

\newpage
\section{Variando a Quantidade de Dados}

Os testes variando a quantidade de dados buscavam avaliar a complexidade dos dois algoritmos,
observando seu comportamento com o aumento na entrada de dados. Para esse teste, foram feitas
execuções com dados uniformes, com quantidade variando de 10$^6$ a 10$^{10}$. 
O algoritmo Quicksort foi executado com 2 partições, e Samplesort manteve a fórmula anteriormente
usada: partições = máquinas x núcleos. As ordenações
foram feitas em cinco máquinas, e o teste foi repetido 5 vezes para cada quantidade de
dados. 
A seguir são apresentados os tempos de ordenação total e relativo a cada conjunto de 10$^6$ dados,
que incluem a média dos tempos para as 5 execuções.


A Figura \ref{fig:DadosTempos} apresenta o tempo médio de ordenação de cada conjunto. O gráfico
apresenta os eixos x e y em escala logarítmica para melhor visualização. Os resultados mostram que
o tempo médio do Quicksort foi ligeiramente menor com o conjunto de 10$^6$ dados, mas nos demais
casos se manteve acima do tempo médio do Samplesort. Além disso, quanto maior o
arquivo, maior foi a diferença entre os tempos. 

\begin{defaultFigure}{0.75\textwidth}{graficos/DadosTempo.pdf}
{Gráfico dos tempos médios para ordenação de 10$^6$ a 10$^{10}$ dados em 5 máquinas}
\label{fig:DadosTempos}
\end{defaultFigure}
 

<INCLUIR DESCRIÇÃO DA TABELA>
\begin{defaultTable}{| B R{5}}
{
\multirow{2}{*}{Dados} & \multicolumn{2}{T}{Tempo (s) Quicksort}  & \multicolumn{2}{T}{Tempo (s)
Samplesort}  \\\cline{2-5}
\rowstyle{\bfseries} 
			&	Média	&	COV		&	Média	&	COV		\\ \hline \hline
$10^6$		&	18		&	0.001	&	20		&	0.004	\\ \hline
$10^7$		&	34		&	0.025	&	30		&	0.028	\\ \hline
$10^8$		&	212		&	0.017	&	144		&	0.010	\\ \hline
$10^9$		&	2246	&	0.036	&	1356	&	0.027	\\ \hline
$10^{10}$	&	30246	&	0.085	&	14816	&	0.005	\\ \hline


}
{Tempos estatísticos para ordenação de diferentes quantidades de dados em 5 máquinas}
\label{tab:DadosTempo}
\end{defaultTable}


Com a finalidade de dimensionar o \textit{overhead} do algoritmo, isto é, a sobrecarga ou custo
adicional de comunicação entre os processadores mestre e escravos, assim como sua escalabilidade em
relação ao número de dados, foi calculado o tempo médio relativo de ordenação para cada conjunto de
10$^6$ dados. 
 
<INLCUIR DESCRIÇÃO GRÁFICO MEGADADOS>
 
\begin{defaultFigure}{0.75\textwidth}{graficos/DadosMegaDados.pdf}
{Gráfico do tempo médio relativo para ordenação de conjuntos de 10$^6$ dados em 5 máquinas}
\label{fig:DadosMegaDados}
\end{defaultFigure}
 
 
% DESCRIÇÃO GRÁFICO COMPLEXIDADE
% 
<INCLUIR GRÁFICO COMPLEXIDADE>
% 
% \newpage


% --------------------------------------------------------
% -------------------- MÁQUINAS --------------------------
% --------------------------------------------------------
\newpage
\section{Variando a Quantidade de Máquinas}

Os testes variando a quantidade de máquinas foram realizados para avaliar a escalabilidade do
algoritmo, de forma a medir o ganho de tempo obtido com o aumento do número de máquinas
utilizadas na ordenação. Para tal, cada algoritmo foi executado cinco vezes e foi medido o tempo
gasto na ordenação. 
Além do tempo gasto em cada execução, foram calculados também o \textit{speedup}
e a eficiência de cada algoritmo. Cada teste foi realizado com um conjunto de 10$^8$
chaves uniformes, que foram ordenadas por grupos de 2, 3, 4 e 5 máquinas.

A Figura \ref{fig:MaquinasTempos} representa graficamente a variação do tempo à medida que o número
de máquinas aumenta. Como se pode observar, a variação, representada pela inclinação da curva é
similar para os dois algoritmos, mantendo uma queda gradativa no tempo de execução com o aumento
do número de máquinas. Ainda pode-se observar que a diminuição do tempo não é linear, não
ocorre na mesma proporção que o aumento no número de máquinas. 


\begin{defaultFigure}{0.75\textwidth}{graficos/MaquinasTempo.pdf}
{Gráfico dos tempos médios para ordenação de 10$^8$ dados em 2 a 5 máquinas}
\label{fig:MaquinasTempos}
\end{defaultFigure}

As Tabelas \ref{tab:MaquinasTempoS} e \ref{tab:MaquinasTempoQ} apresentam os tempos médios, em
segundos,
necessários para as ordenações de $10^8$ elementos em diferentes quantidade de máquinas, com os
algoritmos Quicksort e Samplesort, respectivamente. 

O tempo médio para a execução em duas máquinas do algoritmo Samplesort foi de aproximadamente 6
minutos, do algoritmo Quicksort foi perto de 7 minutos. Em três máquinas o algoritmo Samplesort
gastou cerca de 4 minutos para concluir a ordenação, 36\% menos que o Quicksort, que precisou
de pouco mais de 5 minutos.
Com quatro máquinas os tempos de execução para Samplesort e Quicksort foram de aproximadamente 3 e 4
minutos, respectivamente.
Em cinco máquinas a ordenação foi realizada em aproximadamente 4 minutos
com Quicksort e 2 minutos com Samplesort. De forma geral, a avaliação dos resultados mostra que
para cada acréscimo de uma máquina houve uma diminuição de 1 minuto no tempo de execução do
algoritmo.

\begin{defaultTable}{| B R{6}}
{
\multirow{2}{*}{Máquinas} & \multicolumn{6}{T}{Tempo (s) Samplesort}  \\\cline{2-7}
\rowstyle{\bfseries} 
		&	Menor	&	Maior	&	Média	&	Mediana	&	Desvio Padrão	& 	COV	\\ \hline \hline
2		&	328		&	338	&	334	&	335	&	3,344			&	0,010	\\ \hline
3		&	225		&	230	&	227	&	227	&	1,735			&	0,008	\\ \hline
4		&	175		&	182	&	178	&	177	&	2,518			&	0,014	\\
\hline
5		&	139		&	144	&	142	&	142	&	1,864			&	0,013	\\
\hline
}
{Tempos estatísticos para ordenação com Samplesort de $10^8$ chaves em diferentes quantidades de
máquinas}
\label{tab:MaquinasTempoS}
\end{defaultTable}

\begin{defaultTable}{| B R{6}}
{
\multirow{2}{*}{Máquinas} & \multicolumn{6}{T}{Tempo (s) Quicksort}  \\\cline{2-7}
\rowstyle{\bfseries} 
	&	Menor	&	Maior	&	Média	&	Mediana	&	Desvio Padrão	&	COV	\\ \hline \hline
2	&	399	&	415	&	406	&	404	&	5,450			&	0,013	\\ \hline
3	&	291	&	320	&	309	&	312	&	9,724			&	0,031	\\ \hline
4	&	247	&	267	&	256	&	256	&	7,101			&	0,028	\\ \hline
5	&	215	&	233	&	224	&	220	&	6,938			&	0,031	\\ \hline
}
{Tempos estatísticos para ordenação com Quicksort de $10^8$ chaves em diferentes quantidades de máquinas}
\label{tab:MaquinasTempoQ}
\end{defaultTable}

O tempo médio de execução do algoritmo Quicksort em cinco máquinas foi cerca de 45\% menor que o
tempo médio obtido em duas máquinas. 
Para o Samplesort o percentual reduzido é ainda maior: em cinco máquinas reduz-se 57\% o tempo
necessário para a execução em duas máquinas. 

Em todos os casos, o coeficiente de variação calculado foi menor que 0,03, o que indica
tempos de execução bastante homogêneos para os dois algoritmos. 




A fim de avaliar o desempenho do algoritmo nas diferentes configurações de máquinas foram calculadas
as métricas \textit{speedup}
e eficiência.

Como foi descrito na Seção \ref{sec:computacaoparalela}, o \textit{speedup} indica quão mais rápida
é a aplicação paralela comparada à aplicação sequencial. Já a eficiência  indica a taxa de
utilização de cada processador. Ela é  excelente quando é 100\%, indicando que os processadores tem
utilização total.
Nesse trabalho o algoritmo foi implementado apenas em paralelo, assim, considera-se para fins de
comparação a execução em quatro processadores (duas máquinas), ao invés da execução sequencial. As
fórmulas
foram adaptadas para verificar a melhora de desempenho e eficiência obtidas a partir de duas
máquinas: 
\[ S = \frac{T_{4 \quad processadores}}{T_{paralelo}}  
\quad\mbox{e}\quad
 E = \frac{S_{real}}{S_{ideal}}\]

Os tempos descritos nas Tabelas \ref{tab:MaquinasTempoS} e \ref{tab:MaquinasTempoQ} foram
utilizados no cálculo do \textit{speedup} e, consequentemente, da eficiência.
A Tabela \ref{tab:MaquinasSpeedup} mosta os valores do \textit{speedup} real calculado e o valor
ideal para comparação. 
Considera-se que a execução com quatro processadores é o valor de referência, e assim
o \textit{speedup} real é igual ao ideal. 
Observa-se que à medida que aumenta o número de processadores o \textit{speedup} real se afasta do
\textit{speedup} ideal. 
A variação do \textit{speedup} já era um comportamento esperado, uma vez que paralelizações em geral
introduzem sobrecargas para realizar o balanceamento e a comunicação entre os processos.

\begin{defaultTable}{|N{25mm}|C{24mm}|C{24mm}|C{12mm}| }
{
\multirow{2}{*}{Processadores} & \multicolumn{2}{T}{Speedup Real} & \multirow{2}{*}{Ideal} \\ \cline{2-3}
\rowstyle{\bfseries} 
	&	Samplesort  &	Quicksort &	\\ \hline \hline
4	&	1			&	1		&	1	\\ \hline
6	&	1,457		&	1,312	&	1,5	\\ \hline
8	&	1,875		&	1,586	&	2	\\ \hline
10	&	2,348		&	1,813	&	2,5	\\ \hline
}
{Resultados do \textit{speedup} para execuções de 4 a 10 processadores}
\label{tab:MaquinasSpeedup}
\end{defaultTable}


A Figura \ref{fig:MaquinasSpeedup} apresenta a curva do \textit{speedup} real dos algoritmos
Quicksort e Samplesort à medida que o número de processadores aumenta, além  da curva de
\textit{speedup} ideal para comparação. 
Como se pode ver, inicialmente todos os valores de \textit{speedup} são iguais mas, com o aumento no número de processadores, a curva representando o valor do \textit{speedup} real se afasta da curva
do \textit{speedup} ideal, fato que ocorre com maior acentuação no algoritmo Quicksort.


\begin{defaultFigure}{0.75\textwidth}{graficos/MaquinasSpeedup.pdf}
{Gráfico do \textit{speedup} para ordenação de 10$^8$ dados em em 2 a 5 máquinas}
\label{fig:MaquinasSpeedup}
\end{defaultFigure}

A Tabela \ref{tab:MaquinasEficiencia} apresenta os valores de eficiência calculados para as
diferentes quantidades de processadores utilizados na ordenação. Observa-se que, conforme aumenta a
quantidade de processadores, o valor da eficiência decresce, mostrando que, com maior número de
processadores, a taxa de utilização é menor. Além disso, observa-se que a eficiência
do algoritmo Samplesort é maior que a eficiência do Quicksort o que indica uma melhor escalabilidade
do algoritmo Samplesort. 

\begin{defaultTable}{|N{25mm}|C{24mm}|C{24mm}| }
{
\multirow{2}{*}{Processadores} & \multicolumn{2}{T}{Eficiência (\%)} \\
\cline{2-3}
\rowstyle{\bfseries} 
	&	Samplesort  &	Quicksort 	\\ \hline \hline
4	&100				&	100			\\ \hline
6	&97,12			&	91,39		\\ \hline
8	&93,97			&	80,82		\\ \hline
10	&94,70			&	74,10		\\ \hline
}
{Resultados da \textit{eficiência} para execuções de 4 a 10 processadores}
\label{tab:MaquinasEficiencia}
\end{defaultTable}



A Figura \ref{fig:MaquinasEficiencia} representa graficamente a eficiência em diferentes números de
processadores. Nela observa-se que, à medida que o número de processadores aumenta o percentual da
eficiência diminui, o que ocorre mais acentuadamente no algoritmo Quicksort. 

\begin{defaultFigure}{0.75\textwidth}{graficos/MaquinasEficiencia.pdf}
{Gráfico da eficiência para ordenação de 10$^8$ dados em em 2 a 5 máquinas}
\label{fig:MaquinasEficiencia}
\end{defaultFigure}

\section{Considerações Finais}
