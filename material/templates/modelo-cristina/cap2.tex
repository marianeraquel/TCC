\chapter{Trabalhos Relacionados} 
\label{cap:trabalhos-relacionados}




Este cap�tulo inclui muitas cita��es bibliogr�ficas. Os principais
itens de bibliografia citados s�o livros, artigos em confer�ncias,
artigos em {\it journals} e p�ginas Web. A bibliografia deve seguir o
padr�o ABNT~\cite{abnt}\footnote{Este n�o � o endere�o oficial da ABNT pois as Normas T�cnicas oficiais s�o pagas e n�o est�o dispon�veis na Web. Consulte a biblioteca.}.

A bibliografia � feita no padr�o {\tt bibtex}.  As refer�ncias s�o
colocadas em um arquivo separado, com termina��o {\tt .bib}, que �
inclu�do no arquivo fonte principal ({\tt .tex}).  Os elementos de
cada item bibliogr�fico que devem constar na bibliografia s�o
apresentados a seguir.

Para livros, o formato da bibliografia no arquivo fonte � o seguinte:

\begin{verbatim}
    @Book{linked,
      author =       {A. L. Barabasi},
      title =        "{Linked: The New Science of Networks}",
      publisher =    {Perseus Publishing},
      year =         {2002},
    }
\end{verbatim}

A cita��o deste livro se faz da seguinte forma \verb#\cite{linked}# e o resultado fica assim~\cite{linked}.
Para os artigos em {\it journals}, veja por exemplo~\cite{acmsurveys},
descrito da seguinte forma no arquivo {\tt .bib}:


\begin{verbatim}
   @article{acmsurveys,
     author = {Deepayan Chakrabarti and Christos Faloutsos},
     title = {Graph mining: Laws, generators, and algorithms},
     journal = {ACM Computing Surveys},
     volume = {38},
     number = {1},
     year = {2006},
     pages = {2-59},
     publisher = {ACM},
     address = {New York, NY, USA},
   }
\end{verbatim}

O artigo~\cite{3faloutsos} foi publicado em confer�ncia.  Embora
�s vezes seja dif�cil distinguir um artigo publicado em {\it
  journal} de um artigo publicado em confer�ncia, esta distin��o �
fundamental.  Em caso de d�vida, procure ajuda de seu orientador.

Veja tamb�m duas cita��es juntas~\cite{rp99,mar00}  e como citar
endere�os Web~\cite{irl:06}. 
O trabalho realizado para editar as cita��es no formato correto �
compensado por uma bibliografia impec�vel.

