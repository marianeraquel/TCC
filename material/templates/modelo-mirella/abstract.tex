
\dica{Caso voc� n�o saiba escrever em ingl�s, procure ajuda para escrever o abstract. Veja esse exemplo de resumo (tirado de \citep{SantosMSV06}), com a descri��o de cada parte em vermelho.}

~
 
\redcomment{(CONTEXTO:)} A Web � abundante em p�ginas que armazenam dados de forma impl�cita.  
\redcomment{(PROBLEMA:)} Em muitos casos, estes dados est�o presentes em 
textos semiestruturados sem a presen�a de delimitadores 
expl�citos e organizados em uma estrutura tamb�m impl�cita.  
\redcomment{(SOLU��O:)} Neste artigo apresentamos uma nova abordagem 
para extra��o em textos semi-estruturados baseada em 
Modelos de Markov Ocultos (Hidden Markov Models - HMM).  
\redcomment{(ESTADO-DA-ARTE e M�TODO PROPOSTO:)} Ao contr�rio 
de outros trabalhos baseados em HMM, nossa abordagem d� 
�nfase � extra��o de metadados al�m dos dados propriamente 
ditos. Esta abordagem consiste no uso de uma estrutura aninhada 
de HMMs, onde um HMM principal identifica os atributos no texto 
e HMMs internos, um para cada atributo, identificam os dados e 
metadados. Os HMMs s�o gerados a partir de um treinamento 
com uma fra��o de amostras da base a ser extra�da. 
\redcomment{(RESULTADOS:)} Nossos  experimentos com an�ncios de 
classificados retirados da Web mostram que o processo de 
extra��o alcan��veis de qualidade acima de 0,97 com a medida F, 
mesmo se esta fra��o de treinamento � pequena. \\

~

\dica{Na monografia, esse resumo pode ser um pouco maior, talvez adicionando mais uma ou duas frases a cada um dos �tens.}

\keywords{Keyword1, Keyword2, Keyword3}
