\chapter{Conclus�o}\label{conclusao}

Este documento apresentou resumidamente os passos principais para escrever uma monografia (gradua��o ou p�s-gradua��o). Espera-se que os estudantes que tenham lido at� aqui, tenham total condi��o de come�ar a escrever seu pr�prio trabalho. A maioria dos exemplos utilizados � real, o que possibilita o acesso aos textos originais (a partir das suas refer�ncias). Finalmente, a dica \textit{default} de buscar exemplos na Web continua valendo. Na busca de textos completos de disserta��es e teses, sugere-se busc�-las nos sites das bibliotecas das universidades. 

~

\dica{
A conclus�o pode ser mais espec�fica que a introdu��o, pois o 
trabalho j� foi descrito. Nesse modo, � importante informar (um 
par�grafo/linha por item): (1) resumo do que o trabalho apresentou,  
(2) principais resultados e contribui��es,  
(3) coment�rios sobre a import�ncia, relev�ncia ou  
(4) dicas para o uso pr�tico do seu trabalho (como os 
resultados dos experimentos podem ajudar na pr�tica...), e
(5) trabalhos futuros (evite entregar suas ideias de 
trabalhos mais inovadores de gra�a).
}

Um bom exemplo de conclus�o � apresentado a seguir, retirado de artigo cient�fico  \cite{RaghavanVRYS07}:
 
\begin{Exemplo}
As cloud-based services transition from marketing vaporware 
to real, deployed systems, the demands on traditional Webhosting and Internet service providers are likely to shift 
dramatically.  In particular, current models of resource 
provisioning and accounting lack the flexibility to effectively 
support the dynamic composition and rapidly shifting load 
enabled by the software as a service paradigm.  We have 
identified  one key aspect of this problem, namely the need to 
rate limit network traffic in a distributed fashion, and provided 
two novel algorithms to address this pressing need.
Our experiments show  that naive implementations based on 
packet arrival information are unable to deliver adequate levels 
of fairness, and, furthermore, are unable to cope with the 
latency and loss present in ... 
Our results demonstrate  that it is possible to recreate, at 
distributed points in the network, the flow behavior that end 
users and network operators expect from a single centralized 
rate limiter. Moreover, it is possible ... 
\end{Exemplo}

Finalmente, a conclus�o (dependendo do tamanho) tamb�m pode ser dividida em se��es, como a da tese de doutorado de \cite{Cordeiro2011}:

\begin{Exemplo}
7 Conclusion \\
7.1 Main Contributions of this Ph.D. Work \\
7.1.1 The Method Halite for Correlation Clustering \\
7.1.2 The Method BoW for Clustering Terabyte-scale Datasets \\
7.1.3 The Method QMAS for Labeling and Summarization \\
7.2 Discussion  \\
7.3 Difficulties Tackled  \\
7.4 Future Work \\
7.4.1 Using the Fractal Theory and Clustering Techniques to Improve the
Climate Change Forecast  \\
7.4.2 Parallel Clustering  \\
7.4.3 Feature Selection by Clustering  \\
7.4.4 Multi-labeling and Hierarchical Labeling \\
7.5 Publications Generated in this Ph.D. Work 
\end{Exemplo}

