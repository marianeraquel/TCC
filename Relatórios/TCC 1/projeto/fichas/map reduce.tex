map reduce:
\begin{itemize}
\item map reduce como proposta para processamento rápido e fácil de grandes quantidades de dados
\item com o map reduce, é possível ordenar de forma rápida e fácil
\end{itemize}

Hadoop, the most well-known open source implementation of MapReduce and a proven commodity easily available to academia.

it becomes clear that the MapReduce programming model is a powerful abstraction that separates the what from the how of data-intensive processing. What’s more revealing and interesting rare the “Big Ideas” that the authors summarize for MapReduce:

scale out, not up
assumir que falhas são comuns 
mover o processamento para os dados
processar os dados sequencialmente e evitar o acesso aleatório
esconder detalhes de implementação no nível do sistema do desenvolvedor
escalabilidade 


 The focus is on controlling code execution and data flow—to design algorithms that are both scalable and efficient. 
 
 [Lin and Dyer 2010]
 
 
 Unstructured data is the largest and fastest growing portion of most enterprise’s assets, often representing 70\% to 80\% of online data. These steep increase in volume of information be- ing produced often exceeds the capabilities of existing commer- cial databases. MapReduce and its open-source implementa- tion Hadoop represent an economically compelling alternative that offers an efficient distributed computing platform for han- dling large volumes of data and mining petabytes of unstruc- tured information.
 
 Currently, there is no available methodology to easily answer this question, and the task of estimating required resources to meet application performance goals is the solely user’s respon- sibility. The users need to perform adequate application test- ing, performance evaluation, capacity planning estimation, and then request appropriate amount of resources from the service provider. 

 One of the major Hadoop benefits is its ability of dealing with failures (disk, processes, node failures) and allowing the user job to complete. 


Facebook, Yahoo!, and eBay process terabytes of data and event logs per day on their Hadoop clusters for spam detection, business intel- ligence and different types of optimization. 

 
 [Cherkasova 2011]
 
 
 
 Hadoop, a implementação fonte mais conhecida aberto de MapReduce e uma mercadoria comprovada facilmente disponíveis para a academia.

torna-se claro que o modelo de programação MapReduce é uma poderosa abstração que separa o. que da como de dados de processamento intensivo O que é mais revelador e interessante raros os "Big Ideas" que os autores resumem para MapReduce:

dimensionar, não se
Que São assumir falhas comuns
mover o parágrafo OS Processamento Dados
processar OS Dados sequencialmente e evitar o Acesso aleatório
Detalhes de implementação esconder nenhum Nível Sistema não fazer desenvolvedor
escalabilidade


 O foco é sobre como controlar a execução de código e de fluxo de dados para algoritmos de design que são escaláveis ​​e eficientes.
 
 [Lin e Dyer 2010]
 
 
 Dados não estruturados é a porção maior e mais rápido crescimento dos ativos mais empresa, muitas vezes, representando 70% \ 80% \ de dados on-line. Estes aumento acentuado no volume de informação ser-ing produzido muitas vezes ultrapassa as capacidades existentes de comer-ciais bancos de dados. MapReduce e seu código-fonte aberto imple-ção Hadoop representam uma alternativa economicamente atraente, que oferece uma plataforma de computação distribuída eficiente para han-manuseamento adversas grandes volumes de dados e petabytes de mineração de unstruc-estruturado informações.
 
 Atualmente, não existe uma metodologia disponível facilmente responder a esta pergunta, ea tarefa de estimar os recursos necessários para cumprir as metas de desempenho do aplicativo é o usuário apenas de responsabilidade. Os usuários precisam para realizar adequada aplicação de teste-ing, avaliação de desempenho, a estimativa de planejamento de capacidade, e em seguida, solicitar quantidade adequada de recursos do provedor de serviços.

 Um dos principais benefícios do Hadoop é a sua capacidade de lidar com falhas (disco, processos, falhas de nós) e permitindo que o trabalho do usuário para ser concluído.


Facebook, Yahoo!, eBay e terabytes de processo de dados e logs de eventos por dia em seus clusters Hadoop para detecção de spam, negócios intel ligence e diferentes tipos de otimização.

 
 [Cherkasova 2011]
 
 
 
O MapReduce\citep{Dean:2008}  é um modelo de programação paralela criado pela Google para processamento de grandes volumes de dados em \textit{clusters}. Esse modelo propõe simplificar a computação paralela e ser de fácil uso, abstraindo conceitos complexos da paralelização - como tolerância a falhas, distribuição de dados e balanço de carga - e utilizando duas funções principais: map e reduce. A complexidade do algoritmo paralelo não é vista pelo desenvolvedor, que pode se ocupar em desenvolver a solução proposta. O Hadoop \citep{Hadoop:2010} é uma das implementações do MapReduce, um \textit{framework open source } desenvolvido por Doug Cutting em 2005 que provê o gerenciamento de computação distribuída. %Foi, e é amplamente apoiadO e utilizado pela Yahoo!.


% Com grande processamento de dados
MapReduce e sua implementação \textit{open source} Hadoop representam uma alternativa economicamente atraente oferecendo uma plataforma eficiente de computação distribuída para lidar com grandes volumes de dados e mineração de petabytes de informações não estruturadas \citep{Cherkasova:2011}.

Devido ao grande número de algoritmos de ordenação paralela e variadas arquiteturas paralelas, estudos experimentais assumem uma importância crescente na avaliação e seleção de algoritmos apropriados para multiprocessadores.
