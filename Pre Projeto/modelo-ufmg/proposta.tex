\chapter{Proposta do Projeto}

\section{Tema do trabalho}

A ordenação  \\

A computação paralela ...


\section{Relevância}

Uso crescente de computação paralela para sistemas computacionais.

\section{Objetivos}

O projeto propõe a 

(i) Estudar a programação paralela, seus algoritmos e suas possibilidades de implementação em ambiente paralelo multicore;
(ii) Implementar e avaliar o desempenho de um algoritmo de ordenação paralela;
(iii) Estudar e implementar a solução no modelo MapReduce, com o software Hadoop;
(iv) Comparar os resultados obtidos com os algoritmos de ordenação e analisar seu desempenho com relação à quantidade de dados a serem ordenados, variabilidade dos dados de entrada e número máquinas utilizadas.

\section{Resultados esperados}

Com a realização do trabalho, busca-se ampliar e consolidar conhecimentos adquiridos na área de computação paralela, assim como
a capacidade de análise e desenvolvimento de algoritmos paralelos. 
Os resultados esperados incluem entendimento do modelo MapReduce, implementação de algoritmos de ordenação paralela em ambiente Hadoop, assim como testes para comparação de desempenho entre dois ou algoritmos. 

\section{Metodologia}

vamos estudar a importancia da computacao paralela. da ordenaçao.
Vamos estudar o map reduce. do hadoop.
implementar. testar. melhorar se possível. refazer os testes comparando implementacoes e analisar os resultados. 

\section{Infraestrutura necessária}

A infra estrutura necessária ao desenvolvimento do projeto será fornecida pelo Laboratório de Redes e Sistemas (LABORES) do Departamento de Computação (DECOM). Esse laboratório possui um cluster formado por cinco máquinas Dell Optiplex 380, que serão utilizadas na realização dos testes dos algoritmos. Os algoritmos serão desenvolvidos em linguagem Java, de acordo com o modelo MapReduce, no ambiente Hadoop. 

O cluster a ser utilizado apresenta as seguintes características:
\begin{itemize}
\item 5 nodos
\item Processador Intel Core 2 Duo de 3.0 GHz
\item Disco rígido SATA de 500 GB 7200 RPM
\item Memória RAM de 4 GB
\item Placa de rede Gigabit Ethernet
\item Sistema operacional Linux Ubunbu 10.04 32 bits (kernel 2.6.\textbf{XX})
\item Sun Java JDK 1.6.0 19.0-b09 
\item Hadoop 0.20.2
\end{itemize}

\section{Cronograma de trabalho}

\begin{table}
\begin{center}
\renewcommand{\arraystretch}{1.5}
\setlength\tabcolsep{7pt}

\begin{tabular}{|p{2cm} p{11.5cm}|}
\hline
Fevereiro & Definição do tema de trabalho\\ 
Março & Pesquisa bibliográfica e escrita da proposta de projeto\\
Abril & Pesquisa bibliográfica e início da escrita do projeto\\
Maio & Aplicação da metodologia proposta no desenvolvimento de algoritmo e realização de testes \\
Junho & Finalização da escrita do projeto e entrega \\
Julho & Desenvolvimento e testes de algoritmos paralelos\\
Agosto & Desenvolvimento e testes dos algoritmos desenvolvidos \\
Setembro & Análise dos resultados obtidos até o momento, em busca de pontos de melhorias no projeto \\
Outubro & Teste final dos algoritmos, análise e escrita dos resultados \\
Novembro & Finalização da escrita do projeto final, entrega e apresentação \\
\hline
\end{tabular}

\caption{Cronograma proposto para o projeto}
\end{center}
\label{tab:valores}
\end{table}


\cite{Kale:2010} \\
\cite{Manferdelli:2008} \\
\cite{Dean:2008} \\
\cite{Asanovic:2009}`