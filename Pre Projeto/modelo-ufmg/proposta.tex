\chapter{Proposta do Projeto}

\section{Introdução}


A ordenação é o processo de reordenar uma sequência de entrada e produzir uma saída ordenada de acordo com um atributo. 
A ordenação paralela é o processo de uso de múltiplas unidades de processamento para ordenar em conjunto uma sequência. Na criação de algoritmos de ordenação paralela, é ponto fundamental ordenar coletivamente os dados de cada processo individual, de forma a utilizar todas as unidades de processamento e minimizar os custos de redistribuição de chaves entre os processadores. \cite{Kale:2010} 


Fatores como movimentação de dados, balanço de carga, latência de comunicação e distribuição inicial das chaves são considerados ingredientes chave para o bom desempenho da ordenação paralela, e variam de acordo com o algoritmo escolhido como solução. O presente projeto propõe a análise de desempenho, em termos do número de dados a serem ordenados e de máquinas utilizadas, de um ou mais algoritmos de ordenação paralela, implementados de acordo com o modelo MapReduce no ambiente Hadoop.


\section{Relevância}

A arquitetura paralela é um conceito conhecido há várias décadas 

%
%A arquitetura paralela foi construída há muito tempo, mas somente recentemente po- demos experimentá-la a custo razoável, ampliando seu acesso. A literatura indica que estamos experimentando as primeiras versões de arquiteturas multicore e a tendência é de que elas evoluam cada vez mais, marcando o início de uma era de computadores cada vez mais poderosos. O estudo da programação paralela torna-se fundamental neste con- texto, pois essa é a única forma de aproveitar toda a e􏰁ciência permitida por tais má- quinas


Com a recente mudança de foco no desenvolvimento de algoritmos sequenciais para paralelos, conhecer os modelos de programação paralela se tornou uma grande necessidade na computação. Os algoritmos paralelos ainda são um ramo pouco explorado, devido a  maior complexidade no desenvolvimento e recentes aplicações em sistemas \textit{multicore}.


Uso crescente de computação paralela para sistemas computacionais gera a necessidade de algoritmos de ordenação inovadores, desenvolvidos para dar suporte a essas aplicações. 
O MapReduce\cite{Dean:2008}  é um modelo de programação paralela criado pela Google para processamento e geração de grandes volumes de dados em \textit{clusters}. Esse modelo propõe simplificar a computação paralela e ser de fácil uso, abstraindo conceitos complexos da paralelização - como tolerância a falhas, distribuição de dados e balanço de carga - e utilizando duas funções principais: map e reduce. A complexidade do algoritmo paralelo não é vista pelo desenvolvedor, que pode se ocupar em desenvolver a solução proposta. O Hadoop \cite{Hadoop:2010} é uma implementação do MapReduce, um \textit{framework} open source que provê o gerenciamento de computação distribuída. 


\section{Objetivos}

O projeto propõe a 

(i) Estudar a programação paralela, seus algoritmos e suas possibilidades de implementação em ambiente paralelo multicore;
(ii) Implementar e avaliar o desempenho de um algoritmo de ordenação paralela;
(iii) Estudar e implementar a solução no modelo MapReduce, com o software Hadoop;
(iv) Comparar os resultados obtidos com os algoritmos de ordenação e analisar seu desempenho com relação à quantidade de dados a serem ordenados, variabilidade dos dados de entrada e número máquinas utilizadas.

\section{Resultados esperados}

Com a realização do trabalho, busca-se ampliar e consolidar conhecimentos adquiridos na área de computação paralela, assim como
a capacidade de análise e desenvolvimento de algoritmos paralelos no modelo MapReduce. 


Ao final do trabalho, espera-se obter a implementação de algoritmos de ordenação paralela em ambiente Hadoop e uma análise comparativa de desempenho entre tais algoritmos.

\section{Metodologia}

O início do projeto será destinado ao estudo mais detalhado da computação paralela, em especial os algoritmos de ordenação paralela, dos fatores que influenciam o desempenho de tais algoritmos, o modelo MapReduce e a plataforma Hadoop. O passo seguinte é conhecer detalhadamente o algoritmo paralelo a ser implementado e definir as estratégias para sua implentação ambiente Hadoop. 
O algoritmo implementado deve ser cuidadosamente avaliado para verificar um funcionamento adequado com diferentes entradas e número de máquinas. 
Em seguida, serão realizados experimentos para testes de desempenho dos algoritmos com relação à quantidade de máquinas, quantidade de dados e conjunto de dados.  Os resultados obtidos serão analisados e permitirão comparar a performance dos algoritmos em cada situação. 


\section{Infraestrutura necessária}

A infra estrutura necessária ao desenvolvimento do projeto será fornecida pelo Laboratório de Redes e Sistemas (LABORES) do Departamento de Computação (DECOM). Esse laboratório possui um cluster formado por cinco máquinas Dell Optiplex 380, que serão utilizadas na realização dos testes dos algoritmos. Os algoritmos serão desenvolvidos em linguagem Java, de acordo com o modelo MapReduce, no ambiente Hadoop. 

O cluster a ser utilizado apresenta as seguintes características:
\begin{itemize}
\item 5 nodos
\item Processador Intel Core 2 Duo de 3.0 GHz
\item Disco rígido SATA de 500 GB 7200 RPM
\item Memória RAM de 4 GB
\item Placa de rede Gigabit Ethernet
\item Sistema operacional Linux Ubunbu 10.04 32 bits (kernel 2.6.\textbf{XX})
\item Sun Java JDK 1.6.0 19.0-b09 
\item Hadoop 0.20.2
\end{itemize}

\section{Cronograma de trabalho}

Na Tabela \ref{tab:cronograma} está descrito o cronograma esperado para o desenvolvimento do projeto. Cada atividade foi descrita para se adequar da melhor maneira ao tempo disponível do projeto, mas é possível que o cronograma seja refinado posteriormente, com a inclusão de novas atividades ou redistribuição das tarefas existentes. 

\begin{table}[h]
\begin{center}
\renewcommand{\arraystretch}{1.5}
\setlength\tabcolsep{3pt}

\begin{tabular}{| p{8cm} | p{0.5cm} | p{0.5cm} | p{0.5cm} | p{0.5cm} | p{0.5cm} | p{0.5cm} | p{0.5cm} | p{0.5cm} | p{0.5cm} | p{0.5cm} |}
\hline
Atividade &M &A &M &J &J &A &S &O &N &D \\ \hline \hline
Definição do tema de trabalho. & & & & & & & & & & \\ \hline
Pesquisa bibliográfica sobre o tema.  & & & & & & & & & &\\ \hline
Escrita da proposta de projeto.  & & & & & & & & & & \\ \hline
Estudo mais detalhado dos algoritmos de ordenação paralela,  modelo MapReduce e Hadoop.  & & & & & & & & & &\\ \hline
Escrita da introdução e do referencial teórico do projeto.  & & & & & & & & & &\\ \hline
Configuração do ambiente Hadoop no laboratório.  & & & & & & & & & &\\ \hline
Testes iniciais para conhecer o funcionamento do Hadoop.  & & & & & & & & & &\\ \hline
Realização de experimentos com o algoritmo de ordenação paralela encontrado na literatura ou desenvolvido de acordo com a metodologia proposta.  & & & & & & & & & &\\ \hline
Descrição dos experimentos e da metodologia no texto do projeto.  & & & & & & & & & & \\ \hline
Finalização e entrega do projeto.  & & & & & & & & & &\\ \hline
Desenvolvimento  algoritmos de ordenação. & & & & & & & & & & \\ \hline
Desenvolvimento e testes dos algoritmos desenvolvidos  & & & & & & & & & & \\ \hline
Análise dos resultados obtidos até o momento, em busca de pontos de melhorias no projeto  & & & & & & & & & & \\ \hline
Teste final dos algoritmos, análise e escrita dos resultados  & & & & & & & & & & \\ \hline
Escrita e revisão do projeto final.  & & & & & & & & & &\\ \hline
Entrega e apresentação.  & & & & & & & & & & \\ \hline
Revisão nas observações realizadas pela banca.  & & & & & & & & & &\\ \hline
\hline
\end{tabular}

\caption{Cronograma proposto para o projeto}
\end{center}
\label{tab:cronograma}
\end{table}

%\begin{table}[h]
%\begin{center}
%\renewcommand{\arraystretch}{1.5}
%\setlength\tabcolsep{7pt}
%
%\begin{tabular}{|p{2cm} p{11.5cm}|}
%\hline
%%Fevereiro & Definição do tema de trabalho. \\ 
%Março & Pesquisa bibliográfica e escrita da proposta de projeto. \\
%Abril & Pesquisa bibliográfica, contextualização de algoritmos de ordenação paralelos, familiarização com ambiente Hadoop e com o modelo MapReduce. Escrita dos itens introdução e referencial teórico do projeto. \\
%Maio & Realização de experimentos com o algoritmo de ordenação paralela encontrado na literatura ou desenvolvido de acordo com a metodologia proposta. Descrição dos experimentos e da metodologia no texto do projeto.  \\
%Junho & Finalização e entrega do projeto. \\
%Julho & Desenvolvimento  algoritmos de ordenação \\
%Agosto & Desenvolvimento e testes dos algoritmos desenvolvidos \\
%Setembro & Análise dos resultados obtidos até o momento, em busca de pontos de melhorias no projeto \\
%Outubro & Teste final dos algoritmos, análise e escrita dos resultados \\
%Novembro & Finalização da escrita do projeto final, entrega e apresentação \\
%\hline
%\end{tabular}
%
%\caption{Cronograma proposto para o projeto}
%\end{center}
%\label{tab:valores}
%\end{table}

Citações: \\
\cite{Kale:2010} \\
\cite{Manferdelli:2008} \\
\cite{Dean:2008} \\
\cite{Asanovic:2009}