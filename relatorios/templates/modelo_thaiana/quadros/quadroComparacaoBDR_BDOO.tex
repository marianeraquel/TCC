%	 Modelo de Projeto de disserta��o de mestrado
%
% Centro Federal de Educa��o Tecnol�gica de Minas Gerais - CEFET-MG
% Departamento de Pesquisa e P�s-Gradua��o - DPPG
% Autor: Grupo de Pesquisas em Sistemas Inteligentes (GPSI)
%
% Parte: Tabela "Comparativo dos Bancos de Dados Relacionais e Orientados a Objetos"

\begin{table}[h]
  \begin{center}
    \caption{Hierarquia de restri��es das quest�es.}
    \vspace{0.5cm}
    \begin{tabular}{p{7cm}|p{7cm}}
      \hline \hline
      \begin{center}\textbf{BD Relacionais} \end{center} & \begin{center}\textbf{BD Orientados a Objetos}\end{center} \\
      \hline \hline
      Os dados s�o passivos, ou seja, certas opera��es limitadas podem ser automaticamente acionadas quando os dados s�o usados. Os dados s�o ativos, ou seja, as solicita��es fazem com que os objetos executem seus m�todos. \\
      \hline
      Os processos que usam dados mudam constantemente. \\
      \hline \hline
    \end{tabular}
    \vspace{0.2cm}
    \\ FONTE - CARVALHO, 2001. p. 12.
  \end{center}
\end{table}
