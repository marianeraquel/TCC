dados: 
\begin{itemize}
\item crescimento dos dados
\item grandes dados: apenas disco
\item grandes dados: problematização (tempo, limite de memória)
\item grandes dados: sort benchmark
\end{itemize}

 in the past decade, the amount of data available has increased by several orders of magnitude. This clearly poses serious challenges in terms of computation for traditional text processing approaches using a single computer. As a result, efficient distributed computing has become more crucial than ever.
 
 first give convincing facts about the importance and availability of large data corpora. It follows naturally that large-scale data processing using computer clusters is inevitable.
 
 

Na última década, a quantidade de dados disponíveis aumentou várias ordens de grandeza, fazendo o processamento dos dados um desafio para a computação sequencial. Como resultado, torna-se crucial substituir a computação tradicional por computação distribuída eficiente. 

É um caminho natural para o processamento de dados em larga escala o uso de clusters.
  
[Lin and Dyer 2010]

O desenvolvimento de soluções capazes de lidar com grandes volumes de dados é uma das preocupações atuais, tendo em vista a quantidade de dados processados diariamente, e o rápido crescimento desse volume de dados.
Não é fácil medir o volume total de dados armazenados digitamente, mas uma estimativa da IDC colocou o tamanho do "universo digital" em 0,18 zettabytes em 2006, e previa um crescimento dez vezes até 2011 (para 1,8 zettabytes).
 \textit{The New York Stock Exchange} gera cerca de um terabyte de novos dados comerciais por dia. O Facebook armazena aproximadamente 10 bilhões de fotos, que ocupam mais de um petabyte. \textit{The Internet Archive} armazena aproximadamente 2 petabytes de dados, com aumento de 20 terabytes por mês
\citep{Hadoop:2010}. Estima-se que dados não estruturados são a maior porção e de a mais rápido crescimento dentro das empresas, o que torna o processamento de tal volume de dados muitas vezes inviável.
