\chapter{Referencial Teórico}
\label{cap:referencial}

\begin{itemize}
\item computação paralela
\item modelos de computação paralela (memória compartilhada, distribuida, threads, paralelismo de dados e map reduce)
\item map reduce
\item ordenação
\end{itemize}


% computação paralela


\chapter{Ordenação Paralela}
\label{cap:ordenacao}

\begin{itemize}
\item importância da ordenação paralela
\item formas de ordenação: memória e disco
\item grandes dados: apenas disco
\item grandes dados: problematização (tempo, limite de memória)
\item grandes dados: sort benchmark
\end{itemize}


\begin{itemize}
\item algoritmos de ordenação paralelos
\item funcionamento geral 
\item condições / ingredientes / limites
\item diferentes algoritmos para diferentes aplicações
\item descrição de algoritmos (e diagramas): sample sort, quick sort

\end{itemize}
\section{Algoritmos de ordenação Paralela}

\textbf{Condições de implementação de algoritmos paralelos de ordenação}

\paragraph*{ $\bullet$ Habilidade de explorar distribuições iniciais parcialmente ordenadas}
Alguns algoritmos podem se beneficiar de cenários nos quais a sequência de entrada dos dados é mesma, ou pouco alterada. Nesse caso, é possível obter melhor desempenho ao realizar menos trabalho e movimentação de dados. 
Se a alteração na posição dos elementos da sequência é pequena o suficiente, grande parte dos processadores mantém seus dados iniciais e precisa se comunicar apenas com os processadores vizinhos.


\paragraph*{Movimentação dos dados}
A movimentação de dados entre processadores deve ser mínima durante a execução do algoritmo. Em um sistema de memória distribuída, a quantidade de dados a ser movimentada é um ponto crítico, pois o custo de troca de dados pode dominar o custo de execução total e limitar a escalabilidade.


\paragraph*{Balanceamento de carga} 
O algoritmo de ordenação paralela deve assegurar o balanceamento de carga ao distribuir os dados entre os processadores. Cada processador deve receber uma parcela equilibrada dos dados para ordenar, uma vez que o tempo de execução da aplicação é tipicamente limitada pela execução do processador mais sobrecarregado. 

\paragraph*{Latência de comunicação}
A latência de comunicação é definida como o tempo médio necessário para enviar uma mensagem de um processador a outro. 
Em grandes sistemas distribuídos, reduzir o tempo de latência se torna muito importante. 

\paragraph*{Sobreposição de comunicação e computação}
Em qualquer aplicação paralela, existem tarefas com focos em computação e comunicação. A sobreposição de tais tarefas permite que sejam feitas tarefas de processamento e ao mesmo tempo operações de entrada e saída de dados, evitando que os recursos fiquem ociosos durante o intervalo de tempo necessário para a transmissão da carga de trabalho. 

\subsection{Sample Sort}
\subsection{Quick Sort}

