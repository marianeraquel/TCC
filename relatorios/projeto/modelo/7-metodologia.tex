\chapter{Metodologia}
\label{cap:metodologia}


O início do projeto será destinado ao estudo mais detalhado da computação paralela, em especial os algoritmos de ordenação paralela, dos fatores que influenciam o desempenho de tais algoritmos, o modelo MapReduce e a plataforma Hadoop. O passo seguinte é conhecer detalhadamente o algoritmo paralelo a ser implementado e definir as estratégias para sua implentação ambiente Hadoop. 
O algoritmo implementado deve ser cuidadosamente avaliado para verificar um funcionamento adequado com diferentes entradas e número de máquinas. 

Em seguida, serão realizados experimentos para testes de desempenho dos algoritmos com relação à quantidade de máquinas, quantidade de dados e conjunto de dados.  Os resultados obtidos serão analisados e permitirão comparar a desempenho dos algoritmos em cada situação. 


\section{Infraestrutura necessária}

A infra estrutura necessária ao desenvolvimento do projeto será fornecida pelo Laboratório de Redes e Sistemas (LABORES) do Departamento de Computação (DECOM). Esse laboratório possui um \textit{cluster} formado por cinco máquinas Dell Optiplex 380, que serão utilizadas na realização dos testes dos algoritmos. Os algoritmos serão desenvolvidos em linguagem Java, de acordo com o modelo MapReduce, no ambiente Hadoop. 

Cada máquina do \textit{cluster} apresenta as seguintes características:
\begin{packed_enum}
\item Processador Intel Core 2 Duo de 3.0 GHz
\item Disco rígido SATA de 500 GB 7200 RPM
\item Memória RAM de 4 GB
\item Placa de rede Gigabit Ethernet
\item Sistema operacional Linux Ubunbu 10.04 32 bits %(kernel 2.6.\textbf{XX})
\item Sun Java JDK 1.6.0 19.0-b09 
\item Hadoop 0.20.2
\end{packed_enum}


\section{Cronograma de trabalho}


O cronograma de trabalho inclui as atividades que devem ser realizadas e como elas devem ser alocadas durante as disciplinas TCC I e TCC II para que o projeto possa ser concluído com sucesso.
As tarefas a serem desenvolvidas estão descritas a seguir:

\begin{num_enum}
 \item \label{c1} Pesquisa bibliográfica sobre o tema do projeto e escrita da proposta
 \item \label{c2} Estudo mais detalhado dos algoritmos de ordenação paralela,  modelo MapReduce e Hadoop.
 \item \label{c3} Configuração do ambiente Hadoop no laboratório.
 \item \label{c4} Implementação e testes.
 \item \label{c5} Escrita, revisão e entrega do relatório. 
 \item \label{c7} Análise comparativa entre os resultados.
 \item \label{c8} Escrita e revisão do projeto final.
 \item \label{c9} Entrega e apresentação.
 \end{num_enum}
 
 
Na Tabela \ref{tab:cronograma} está descrito o cronograma esperado para o desenvolvimento do projeto. Cada atividade foi alocada para se adequar da melhor maneira ao tempo disponível, mas é possível que o cronograma seja refinado posteriormente, com a inclusão de novas atividades ou redistribuição das tarefas existentes. 

\begin{table}[h]

\renewcommand{\arraystretch}{1}
\setlength\tabcolsep{3pt}
\begin{center}
\begin{tabular}{| c | c | c | c | c | c | c | c | c | c | c |}
\hline

Atividade &Fev &Mar &Abr &Mai &Jun &Jul &Ago &Set &Out &Nov \\ \hline \hline
\ref{c1}   &$\bullet$ &$\bullet$ & & & & & & & & \\ \hline
\ref{c2}   & &$\bullet$ &$\bullet$ & & & & & & & \\ \hline
\ref{c3}   & & &$\bullet$ & & & & & & & \\ \hline
\ref{c4}   & & &$\bullet$ &$\bullet$ & &$\bullet$ &$\bullet$ & & & \\ \hline
\ref{c5}   & & & &$\bullet$ &$\bullet$ & & & & & \\ \hline
\ref{c7}   & & & & & & & &$\bullet$ & & \\ \hline
\ref{c8}   & & & & & & & & &$\bullet$ & \\ \hline
\ref{c9}   & & & & & & & & & &$\bullet$ \\ 
\hline
\end{tabular}
\end{center}
\caption{Cronograma proposto para o projeto}
\label{tab:cronograma}
\end{table}

\section{Desenvolvimento //}

\section{Descrição dos experimentos}