\chapter{Resultados Preliminares}
\label{cap:resultados}

Neste capítulo são apresentados e analisados os resultados obtidos nessa fase do projeto, de acordo com os testes descritos anteriormente.

\section{Benchmarks: TeraSort e Sort}

A Tabela  \ref{tab:TeraSort} apresenta os resultados obtidos para o \textit{benchmarks} TeraSort. 

\begin{table}[htbp]
\caption{Resultados do benchmark TeraSort para execução em 2 máquinas}
\begin{center}
\begin{tabular}{|l|ccc|} \hline
Algoritmo 		&Tempo (seg)	 	&Tarefas Map 	&Tarefas Reduce \\ \hline \hline
TeraGen 			&14			&2					&0						\\ \hline 
TeraSort			&22			&2					&1						\\ \hline 
TeraValidade 	&22			&1					&1						\\ \hline 
\end{tabular}
\end{center}
\label{tab:TeraSort}
\end{table}

A Tabela \ref{tab:Sort} apresenta os resultados obtidos para o \textit{benchmarks} Sort. 

\begin{table}[htbp]
\caption{Resultados do benchmark Sort para execução em 4 máquinas}
\begin{center}
\begin{tabular}{|l|ccc|} \hline
Algoritmo 		&Tempo (seg) 	&Tarefas Map 	&Tarefas Reduce\\ \hline \hline
RandomWriter 	&234				&40					&0						\\ \hline 
Sort					&2242				&640				&7						\\ \hline 
\end{tabular}
\end{center}
\label{tab:Sort}
\end{table}

\section{Algoritmo Ordenação por Amostragem}

Os testes feitos com o algoritmo Ordenação por Amostragem tinham como objetivo reproduzir os resultados encontrados no trabalho feito por Pinhão (2011), e gerar resultados 
que serão utilizados posteriormente na comparação de desempenho dos algoritmos. 
O resultado dos experimentos está separado de acordo com o tipo de teste realizado, com variação do conjunto de dados, da quantidade de dados ordenada e da quantidade de máquinas utilizadas. 

%
%\noindent
%\textbf{Diferentes conjuntos de dados}

\noindent
\textbf{Diferentes quantidades de dados}

A Tabela \ref{tab:QuantidadeDados} apresenta os tempos médios de 6 execuções. 

\begin{table}[htbp]
\caption{Resultados da ordenação diferentes quantidade de dados 4 máquinas}
\begin{center}
\begin{tabular}{|l|c|l|l|} \hline
\multirow{2}{*}{Dados} & \multirow{2}{*}{Tamanho em bytes} & \multicolumn{2}{|c|}{Tempo Médio (seg)} \\
\cline{3-4}
	 & 		& Uniforme 	& Normal	\\ \hline \hline
$10^{6}$ &12MB	 	&39		&41		\\ \hline 
$10^{7}$ &120MB		&51		&56		\\ \hline 
$10^{8}$ &1.2GB	 	&221		&231		\\ \hline 
$10^{9}$ &12GB		&1816		&1893		\\ \hline 
$10^{10}$ &120GB 	&17964		&18165		\\ \hline 
\end{tabular}
\end{center}
\label{tab:QuantidadeDados}
\end{table}

\noindent
\textbf{Diferentes quantidades de máquinas}
