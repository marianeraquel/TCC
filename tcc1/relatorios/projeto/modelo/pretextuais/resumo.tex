\begin{resumo}

A ordenação de dados é um dos problemas mais fundamentais da computação devido à sua aplicação em inúmeras áreas. Um grande número de aplicações possui uma fase de computação intensa, na qual uma lista de elementos deve ser ordenada com base em algum de seus atributos. Embora existam diversos estudos sobre algoritmos de ordenação em arquiteturas paralelas, o desenvolvimento de soluções capazes de lidar com grandes volumes de dados é uma das preocupações atuais. O uso de clusters é um caminho natural para o processamento de dados em larga escala e o modelo programação paralela MapReduce foi criado pela Google para facilitar esse processamento em tais ambientes. Este trabalho busca comparar o desempenho de algoritmos de ordenação implementados no modelo MapReduce, com o software Hadoop.  Os experimentos avaliam a escalabilidade dos algoritmos em relação à quantidade de dados ordenados e de máquinas utilizadas e a influência de diferentes conjuntos de entrada no tempo de ordenação. Os resultados iniciais mostram que o algoritmo Ordenação por Amostragem apresenta boa escalabilidade em relação ao número de dados e máquinas, e que as distribuições das cargas de trabalho pouco influenciam no desempenho. 
    
	\paragraph{Palavras-chave:} Ordenação, Programação Paralela, MapReduce, Hadoop, Escalabilidade
\end{resumo}


\begin{abstract}

\textit{Sorting data is one of the most fundamental problems of computing due to its application in many areas. A large number of applications has a phase of intense computation, in which a list of elements must be ordered based on one of its attributes. Although literature abounds with sorting algorithms in parallel architectures, the development of solutions able to handle large data volumes is one of the current concerns. The use of clusters is a natural way to process large-scale data and MapReduce is a parallel programming model created by Google to facilitate processing in such environments. This work compares the performance of sorting algorithms implemented on MapReduce model using Hadoop software. 
The experiments measuring the scalability of the algorithms relating to the amount of data sorted and machines used, and the influence of different sets of input data in sorting time. Initial results show that SampleSort algorithm is quite scalable  in the number of machines and data, and workloads distributions has small influence on performance.}
	
	\paragraph{Keywords:} \textit{Sorting, Parallel Programming, MapReduce, Hadoop, Scalability.}
\end{abstract}
